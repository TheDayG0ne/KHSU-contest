\begin{problem}{Двухсторонние карточки}{стандартный ввод}{стандартный вывод}{1 секунда}{256 мегабайт}

Есть $n$ карточек, на каждой написано по два числа~--- на лицевой стороне $i$-й карточки написано число $a_i$, а на оборотной ~--- $b_i$.

Все числа с лицевых сторон образуют перестановку размера $n$, то есть каждое число от $1$ до $n$ встречается ровно один раз.

То же самое выполняется и для чисел с оборотных сторон.

Возьмите минимальное количество карточек так, чтобы каждое число от $1$ до $n$ встречалось хотя бы раз среди выбранных карточек, неважно на какой из сторон.

\InputFile
Первая строка содержит целое число $n$~--- количество карточек ($1 \leq n \leq 100\,000$).

Вторая строка содержит $n$ целых чисел $a_i$, разделённых пробелом.

Третья строка содержит $n$ целых чисел $b_i$, разделённых пробелом.

\OutputFile
В первую строку выведите $m$~--- количество взятых карточек.

Во второй строке выведите $m$ целых чисел~--- номера взятых карточек, в любом порядке.

Карточки пронумерованы от $1$ до $n$.

\Examples

\begin{example}
\exmpfile{example.01}{example.01.a}%
\exmpfile{example.02}{example.02.a}%
\end{example}

\end{problem}

