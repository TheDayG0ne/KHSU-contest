\begin{problem}{Лягушки и антилягушки}{стандартный ввод}{стандартный вывод}{1 секунда}{256 мегабайт}

На прямой расположено бесконечное количество камней. Каждый камень соответствует целому числу.

Изначально на этих камнях сидят $n$ лягушек, а именно $i$-я лягушка сидит на камне $x_i$. Известно, что изначально никакие две позиции не совпадают.

Но не все лягушки одинаковые: некоторые из них антилягушки!

В начале каждой секунды каждая обычная лягушка прыгает с камня $x$ на камень $x+1$, а каждая антилягушка с камня $x$ на камень $x-1$ .

При этом, если две лягушки оказались в начале секунды на одном камне, то они аннигилируют, то есть, исчезают и больше никак не влияют на последующий процесс.

Заметим, что лягушки прыгают мгновенно, и не находятся ни в каком виде между камнями.

От Вас требуется определить для каждой лягушки аннигилирует ли она и если да, то через сколько секунд.

\InputFile
Первая строка содержит целое число $n$~--- количество лягушек ($1 \leq n \leq 100\,000$).

Следующие $n$ строк содержат пары чисел $x_i$ и $d_i$~--- начальное положение и тип лягушки ($1 \leq x_i \leq 10^9$). Обычная лягушка обозначается через $d_i = 1$, а антилягушка 
через $d_i = -1$.

Гарантируется, что все $x_i$ различны.

\OutputFile
Выведите $n$ строк: в $i$-й строке должно содержатся время, через которое $i$-я лягушка аннигилирует, или $0$, если этого не произойдёт никогда.

\Examples

\begin{example}
\exmpfile{example.01}{example.01.a}%
\exmpfile{example.02}{example.02.a}%
\end{example}

\end{problem}

